\documentclass{article}

\usepackage[utf8]{inputenc}
\usepackage{amsmath}
\usepackage{amsfonts}
\usepackage[final]{pdfpages} 
\usepackage{cases}

\usepackage[top=2cm, bottom=2cm, left=2cm, right=2cm]{geometry}
\newcommand{\g}[2]{g_{#1}^{#2}}


\begin{document}

\section{First time of reach for a sum of a Markov sum}

\subsection{Defining the problem}

Let $(X_{n})_{n\geq0}$ be a markov chain such as :
\begin{itemize}
\item $\forall \, n\geq0 \quad X_{n} \in \{-1,1\} $ 
\item $\forall \, i\geq0 \quad\ \mathbb{P}( X_{i+1}=1 | X_{i}=1 ) = \mathbb{P}( X_{i+1}=-1 | X_{i}=-1 ) = p = 1-q$
\end{itemize}
Let $(S_{n})_{n\geq0}$ be a sequence such as :
$$ S_{n} = \sum_{k=1}^{n}X_{k} $$
Let $T$ be the first time when $S_{n}$ reaches $-1$, i.e  :
$$ T= \inf \{ n\geq0 , S_{n}=-1   \} $$
Let us define the  probability-generating function of T such as :
$$ g_{s}^{x}(z) = \mathbb{E}(z^{T} | S_{0}=s, X_{0}=x) \quad \forall \ x \in \{-1,1\} \ , \ s \in \mathbb{N} \ , \ z \in [0,1]  $$

\subsection{The probability-generating function dynamic}
Given the transition matrix of $(X_{n})_{n\geq0}$, we can infer that for all $x \in \{-1,1\}$ , $s \in \mathbb{N}$ and $z \in [0,1]$ \footnote{For clarity, we omit to note the dependence of G in z} :

\begin{numcases}
		\strut 
       	\g{s}{1} = z\,(p\,\g{s+1}{1}+ q\,\g{s-1}{-1})\\
       	\g{s}{-1} = z\,(q\,\g{s+1}{1}+ p\,\g{s-1}{-1}) 
\end{numcases}
Thus :
\begin{align}
		\g{s+1}{1} = \frac{1}{zp}\, \g{s}{1} - \frac{q}{p}\, \g{s-1}{-1} \\
		\g{s+1}{1} = \frac{1}{zq}\, \g{s}{-1} - \frac{p}{q}\, \g{s-1}{-1}
\end{align}
Which implies that :
\begin{equation}
		\g{s}{-1} =\frac{q}{p}\, \g{s}{1}+\frac{z\, (p^{2}-q^{2})}{p}\, \g{s-1}{-1}
\end{equation}
Leading to :
\begin{equation}
		\g{s+1}{-1} =  (\frac{1}{z\,p}   + z\,(2 - \frac{1}{p}))\, \g{s}{-1}  -  \g{s-1}{-1} 
\end{equation}
Consequently : 
$$
G_{s+1} = 
\begin{pmatrix} 
\g{s+1}{1} \\ \g{s+1}{-1} \\ \g{s}{1} \\ \g{s}{-1} 
\end{pmatrix}
= 
\begin{pmatrix} 
\frac{1}{z p} & 0 &  0 &  -\frac{q}{p} \\
0 & \frac{1}{z p}   + z(2 - \frac{1}{p}) & 0 & -1 \\
1 & 0 &  0 & 0\\
0 & 1 &  0 & 0
\end{pmatrix}
.
\begin{pmatrix} 
\g{s}{1} \\ \g{s}{-1} \\ \g{s-1}{1} \\ \g{s-1}{-1} 
\end{pmatrix} \\
$$
Noting :

$$
A(z) = 
\begin{pmatrix} 
\frac{1}{z p} & 0 &  0 &  -\frac{q}{p} \\
0 & \frac{1}{z p}   + z(2 - \frac{1}{p}) & 0 & -1 \\
1 & 0 &  0 & 0\\
0 & 1 &  0 & 0
\end{pmatrix}
$$
and by imposing a Dirichlet boundary condition we finally get the following second order system :  

\begin{numcases}
		\strut 
       	 G_{s+1}(z) = A(z). G_{s}(z)\\
       	\g{-1}{1} = \g{-1}{-1} = 0\\
       	\lim_{s \to +\infty} \g{s}{-1} = \lim_{s \to +\infty} \g{s}{1} = 0
\end{numcases}

Autrement:
\begin{numcases}
		\strut 
       	 G_{s+1}(z) = A(z). G_{s}(z)\\
       	G_{0}(z) = \begin{pmatrix} x(z) && y(z) && 1 && 1\end{pmatrix}^T \\ 
       	\lim_{s \to +\infty} A(z)^s . G_0(z) = 0
\end{numcases}


$A(z) $is diagnolisable, therefore there exist a matrix $Q(z)$, and $ \lambda_i \in \mathBB{R}, i = 1..4 $ so that:
$$ A(z) = Q(z) diag(\lambda_i, i =1..4) Q(z)^{-1} $$

A quick analysis using Maple shows that exactly two eigen values have theire absolute value greater than $1$, hence we can write that: 
$ |\lambda_i| < 1$ for $ i = 1, 2$ and $|\lambda_i| \geq 1$ for $i = 3, 4$.

Let $U(z) = Q(z)^-1 G_0(z)$, dirichlet condition in infinity are equivalent to
$$ \lambda_i ^ s . (U(z))_i \stackrel{\infty}{\rightarrow} 0 $$
ie 
\begin{numcases}
		\strut 
       	U(z)_3 = 0 \\
				U(z)_4 = 0
\end{numcases}

these 2x2 system gives $x(z)$ and $y(z)$

\end{document}
